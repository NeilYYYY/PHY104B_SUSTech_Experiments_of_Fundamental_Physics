\documentclass[11pt]{article}

\usepackage[a4paper]{geometry}
\geometry{left=2.0cm,right=2.0cm,top=2.5cm,bottom=2.5cm}

\usepackage{ctex} % 支持中文的LaTeX宏包
\usepackage{amsmath,amsfonts,graphicx,subfigure,amssymb,bm,amsthm,mathrsfs,mathtools,breqn} % 数学公式和符号的宏包集合
\usepackage{algorithm,algorithmicx} % 算法和伪代码的宏包
\usepackage[noend]{algpseudocode} % 算法和伪代码的宏包
\usepackage{fancyhdr} % 自定义页眉页脚的宏包
\usepackage[framemethod=TikZ]{mdframed} % 创建带边框的框架的宏包
\usepackage{fontspec} % 字体设置的宏包
\usepackage{adjustbox} % 调整盒子大小的宏包
\usepackage{fontsize} % 设置字体大小的宏包
\usepackage{tikz,xcolor} % 绘制图形和使用颜色的宏包
\usepackage{multicol} % 多栏排版的宏包
\usepackage{multirow} % 表格中合并单元格的宏包
\usepackage{pdfpages} % 插入PDF文件的宏包
\RequirePackage{listings} % 在文档中插入源代码的宏包
\RequirePackage{xcolor} % 定义和使用颜色的宏包
\usepackage{wrapfig} % 文字绕排图片的宏包
\usepackage{bigstrut,multirow,rotating} % 支持在表格中使用特殊命令的宏包
\usepackage{booktabs} % 创建美观的表格的宏包
\usepackage{circuitikz} % 绘制电路图的宏包

\definecolor{dkgreen}{rgb}{0,0.6,0}
\definecolor{gray}{rgb}{0.5,0.5,0.5}
\definecolor{mauve}{rgb}{0.58,0,0.82}
\lstset{
  frame=tb,
  aboveskip=3mm,
  belowskip=3mm,
  showstringspaces=false,
  columns=flexible,
  framerule=1pt,
  rulecolor=\color{gray!35},
  backgroundcolor=\color{gray!5},
  basicstyle={\small\ttfamily},
  numbers=none,
  numberstyle=\tiny\color{gray},
  keywordstyle=\color{blue},
  commentstyle=\color{dkgreen},
  stringstyle=\color{mauve},
  breaklines=true,
  breakatwhitespace=true,
  tabsize=3,
}

% 轻松引用, 可以用\cref{}指令直接引用, 自动加前缀. 
% 例: 图片label为fig:1
% \cref{fig:1} => Figure.1
% \ref{fig:1}  => 1
\usepackage[capitalize]{cleveref}
% \crefname{section}{Sec.}{Secs.}
\Crefname{section}{Section}{Sections}
\Crefname{table}{Table}{Tables}
\crefname{table}{Table.}{Tabs.}

\setmainfont{Palatino_Linotype}[
  Path = ../Fonts/,
  Extension = .ttf
]
\setCJKmainfont{SimHei}[
  Path = ../Fonts/,
  Extension = .ttf
]
\punctstyle{kaiming}
% 偏好的几个字体, 可以根据需要自行加入字体ttf文件并调用

\renewcommand{\emph}[1]{\begin{kaishu}#1\end{kaishu}}

%改这里可以修改实验报告表头的信息
\newcommand{\studentNum}{00000000}
\newcommand{\name}{我是谁}
\newcommand{\exDate}{2025.03.25}
\newcommand{\weekDay}{二}
\newcommand{\ap}{下午}
%%%%%%%%%%%%%%%%%%%%%%%%%%%

\begin{document}

%若需在页眉部分加入内容, 可以在这里输入
% \pagestyle{fancy}
% \lhead{\kaishu 测试}
% \chead{}
% \rhead{}

\begin{center}
    \LARGE \bf 《\, 基\, 础\, 物\, 理\, 实\, 验\, 》\, 实\, 验\, 报\, 告
\end{center}

\begin{center}
    \emph{学号}\underline{\makebox[6em][c]{\studentNum}}
    \emph{姓名}\underline{\makebox[6em][c]{\name}} 
    \emph{实验日期} \underline{\makebox[8em][c]{\exDate}}
    \emph{星期} \underline{\makebox[2em][c]{\weekDay}}\;\underline{\makebox[3em][c]{\ap}}
    {\noindent}
    \rule[8pt]{17cm}{0.2em}
\end{center}

\begin{center}
    \Large \bf 密立根油滴实验
\end{center}

\section*{一、实验目的}

\begin{enumerate}
    \item 验证电荷的不连续性,并测定元电荷的值。
    \item 学习和理解密立根利用宏观量测量微观量的巧妙设想。
\end{enumerate}

\section*{二、实验仪器}

密立根油滴实验装置

\section*{三、实验原理}

实验研究对象是带电的油滴,基本思想是使油滴处于受力平衡状态。油滴通过喷雾器喷
射进入两块相距为$d$的平行极板之间。油在喷射撕裂成油滴时,一般都是带电的。调节两极板之间的电压$U$,可使油滴悬浮在空中。设油滴的质量为$m$,所带的电量为$q$,两极板间的电压为$U$,则油滴所受重力$mg$,静电力$qE=\dfrac{qU}{d}$。\\
油滴悬浮时,重力与电场力平衡(忽略空气浮力):
\begin{equation}
    q=mg\dfrac{d}{U}
\end{equation}
为了测出油滴所带的电量$q$,除了需测定平衡电压$U$和极板间距离$d$外,还需要测量油滴的质量$m$。这种测量电量的方法叫静态平衡法。

因$m$很小,难直接测量。油滴可视为球状,设密度为$\rho$,油滴的质量$m$可表示为:
\begin{equation}
    m=\rho\dfrac{4}{3}\pi r^3
\end{equation}
而油滴的半径$r$可通过其在重力场中的终极速度求出。平行极板不加电压时,油滴受重力作用而加速下降,由于空气阻力的作用,下降一段距离达到某一速度$v_g$后,阻力$f_r$与重力$mg$平衡,油滴将匀速下降。$v_g$称为终极速度。根据斯托克斯定律,阻力$f_r=6\pi r\eta v$,重力与阻力平衡时:
\begin{equation}
    mg=6\pi r\eta v_g
\end{equation}
其中$\eta$是空气的粘滞系数,$r$是油滴的半径。\\
由$(2)$式和$(3)$式得到油滴的半径
\begin{equation}
    r=\sqrt{\dfrac{9\eta v_g}{2g\rho}}
\end{equation}
当两极板间的电压$U$为零时,设油滴匀速下降的距离为$l$,时间为$t_g$,则
\begin{equation}
    v_g=\dfrac{l}{t_g}
\end{equation}
\begin{equation}
    r=\sqrt{\dfrac{9\eta l}{2gt_g\rho}}
\end{equation}

斯托克斯定律是以连续介质为前提的。对于半径为微米量级的油滴,空气已不能看作连续介质,空气的粘滞系数应作如下修正:
\begin{equation}
    \eta'=\dfrac{\eta}{1+\dfrac{b}{pr}}
\end{equation}
其中$b$是修正常数,$p$为大气压强。公式中包含油滴的半径$r$,但因为它处于修正项中,不需要十分精确,故它仍可以用$(6)$式计算。最后得到理论公式:
\begin{equation}
    q=\dfrac{18\pi d}{\sqrt{2\rho g}}\left[\dfrac{\eta}{1+\dfrac{b}{pr}}\cdot\dfrac{l}{t_g}\right]^{\frac{2}{3}}\cdot\dfrac{1}{U}
\end{equation}
上式就是用静态平衡法测定油滴所带电荷的计算公式。其中,
\begin{center}
    \begin{minipage}{0.5\textwidth} % 设定合适的宽度
        \begin{tabbing}
            电容器极板距离\hspace{2em} \= $d=5.00\,mm$\hspace{7em} \= 大气压强\hspace{3em} \= $p=1.013\times10^5\,Pa$ \\
            油的密度 \> $\rho=981\,kg/m^3$ \> 油滴下落距离 \> $l=1.6\,mm$ \\
            重力加速度 \> $g=9.79\,m/s^2$ \> 油滴半径 \> $r$ (由公式$(6)$计算) \\
            空气粘滞系数 \> $\eta=1.83\times10^{-5}\,kg/(m\cdot s)$ \> 平衡电压 \> $U$ (待测) \\
            粘滞系数修正常数 \> $b=0.00823\,N/m$ \> 下落时间 \> $t_g$ (待测)
        \end{tabbing}
    \end{minipage}
\end{center}

\section*{四、实验内容}

\begin{enumerate}
    \item 仪器调整
    
    调节仪器面板上的三只平衡旋钮,将平行电极板调到水平。打开仪器和显示器开关,按“确认”键,选“平衡法”,进入测量界面。
    \item 测量前练习
    \begin{enumerate}
        \item 熟悉操作按键。
        \item 练习喷油和控制油滴平衡。
    \end{enumerate}
    \item 正式测量
    
    将按键$2$置于工作、按键$3$置于平衡、电压调至$200\;V$左右。向油雾口喷油,调节显微镜旋钮,寻找移动缓慢的油滴,缓慢旋转“电压调节”,使油滴处于悬浮状态。选取适中的油滴:目视直径$\approx1\;mm$。记录此时的平衡电压$U$。将按键$3$切换为“提升”,使油滴上升至顶部网格线,然后将按键$3$切换为“平衡”,使油滴悬浮。然后按下按键$2$,使电压为$0\;V$,油滴匀速下降。当下降到$0$格线时,迅速按下计时按钮,开始计时,待油滴下落至$1.6\;mm$格线,停止计时。记下油滴下落时间$t_g$。同一个油滴测量三次下落时间,共测量五个油滴。
    \item 计算元电荷
    \begin{enumerate}
        \item 根据公式$(6)$和$(8)$,代入已知量,可以得到计算油滴电荷的简化公式:
        $$
        q=\dfrac{1.022\times10^{-14}}{\left[\left(1+0.02193\sqrt{t_g}\right)t_g\right]^{\frac{3}{2}}}\times\dfrac{1}{U}
        $$
        据此计算油滴所带电荷$q_i(i=1,2,3,4,5)$。
        \item 计算油滴所带元电荷个数$n_i$。得到每个油滴电量$q_i$后,用$e$的公认值$1.602\times10^{-19}\;C$去除,四舍五入取整得到每个油滴携带的基本电荷个数$n_i$。油滴的元电荷$e_i=\dfrac{q_i}{n_i}$,对$e_i$取平均,求得元电荷值$\overline{e}$,计算元电荷测量值与公认值的相对误差。
    \end{enumerate}
\end{enumerate}

\section*{五、数据记录}

原始数据见附录

\section*{六、数据处理}

\begin{enumerate}
    \item 计算时间平均值;(见附录数据表中的$\overline{t_g}/s$)
    \item 计算各油滴的电量;
    \begin{align*}
        q_1 &= \dfrac{1.022\times10^{-14}}{\left[\left(1+0.02193\sqrt{t_{g_1}}\right)t_{g_1}\right]^{\frac{3}{2}}}\times\dfrac{1}{U_1} = \dfrac{1.022\times10^{-14}}{\left[\left(1+0.02193\sqrt{17.94}\right)\times17.94\right]^{\frac{3}{2}}}\times\dfrac{1}{196} = 6.01\times10^{-19}\;C \\
        q_2 &= \dfrac{1.022\times10^{-14}}{\left[\left(1+0.02193\sqrt{t_{g_2}}\right)t_{g_2}\right]^{\frac{3}{2}}}\times\dfrac{1}{U_2} = \dfrac{1.022\times10^{-14}}{\left[\left(1+0.02193\sqrt{7.24}\right)\times7.24\right]^{\frac{3}{2}}}\times\dfrac{1}{161} = 2.99\times10^{-18}\;C \\
        q_3 &= \dfrac{1.022\times10^{-14}}{\left[\left(1+0.02193\sqrt{t_{g_3}}\right)t_{g_3}\right]^{\frac{3}{2}}}\times\dfrac{1}{U_3} = \dfrac{1.022\times10^{-14}}{\left[\left(1+0.02193\sqrt{13.67}\right)\times13.67\right]^{\frac{3}{2}}}\times\dfrac{1}{295} = 6.10\times10^{-19}\;C \\
        q_4 &= \dfrac{1.022\times10^{-14}}{\left[\left(1+0.02193\sqrt{t_{g_4}}\right)t_{g_4}\right]^{\frac{3}{2}}}\times\dfrac{1}{U_4} = \dfrac{1.022\times10^{-14}}{\left[\left(1+0.02193\sqrt{14.62}\right)\times14.62\right]^{\frac{3}{2}}}\times\dfrac{1}{266} = 6.09\times10^{-19}\;C \\
        q_5 &= \dfrac{1.022\times10^{-14}}{\left[\left(1+0.02193\sqrt{t_{g_5}}\right)t_{g_5}\right]^{\frac{3}{2}}}\times\dfrac{1}{U_5} = \dfrac{1.022\times10^{-14}}{\left[\left(1+0.02193\sqrt{15.22}\right)\times15.22\right]^{\frac{3}{2}}}\times\dfrac{1}{111} = 1.37\times10^{-18}\;C
    \end{align*}
    \item 计算油滴所带元电荷个数$n_i$;
    \begin{align*}
        n_1&=\dfrac{q_1}{e}=\dfrac{6.01\times10^{-19}}{1.602\times10^{-19}}\approx4\;(\text{个}) \\
        n_2&=\dfrac{q_2}{e}=\dfrac{2.99\times10^{-18}}{1.602\times10^{-19}}\approx19\;(\text{个}) \\
        n_3&=\dfrac{q_3}{e}=\dfrac{6.10\times10^{-19}}{1.602\times10^{-19}}\approx4\;(\text{个}) \\
        n_4&=\dfrac{q_4}{e}=\dfrac{6.09\times10^{-19}}{1.602\times10^{-19}}\approx4\;(\text{个}) \\
        n_5&=\dfrac{q_5}{e}=\dfrac{1.37\times10^{-18}}{1.602\times10^{-19}}\approx9\;(\text{个}) 
    \end{align*}
    \item 计算元电荷测量值、与公认值的相对误差。
    \begin{align*}
        \quad e_1&=\dfrac{q_1}{n_1}=\dfrac{6.01\times10^{-19}}{4}=1.50\times10^{-19}\;C \\
        \quad e_2&=\dfrac{q_2}{n_2}=\dfrac{6.01\times10^{-19}}{4}=1.57\times10^{-19}\;C \\
        \quad e_3&=\dfrac{q_3}{n_3}=\dfrac{6.01\times10^{-19}}{4}=1.53\times10^{-19}\;C \\
        \quad e_4&=\dfrac{q_4}{n_4}=\dfrac{6.01\times10^{-19}}{4}=1.52\times10^{-19}\;C \\
        \quad e_5&=\dfrac{q_5}{n_5}=\dfrac{6.01\times10^{-19}}{4}=1.52\times10^{-19}\;C
    \end{align*}
    \begin{align*}
        \overline{e}&=\dfrac{\sum_{i=1}^{5}e_i}{5} \\
        &=\dfrac{1.50\times10^{-19}+1.57\times10^{-19}+1.53\times10^{-19}+1.52\times10^{-19}+1.52\times10^{-19}}{5} \\
        &=1.53\times10^{-19}\;C \\
        \dfrac{\Delta e}{e}&=\dfrac{|\overline{e}-e|}{e}=\dfrac{|1.52\times10^{-19}-1.602\times10^{-19}|}{1.602\times10^{-19}}=5.119\%
    \end{align*}
\end{enumerate}

\section*{七、误差分析}

\begin{enumerate}
    \item 油滴会做布朗运动,难以确定油滴是否达到平衡,导致平衡电压有误差;
    \item 大量给定的常数带来系统误差;
    \item 油滴受到空气浮力;
    \item 实验者测量下落时间产生误差。
\end{enumerate}

\section*{八、实验结论}

本实验利用静态法测量油滴所带电量,进而测得元电荷为$e=1.53\times10^{-19}\;C$,与公认值的相对误差为$5.119\%$。

\end{document}